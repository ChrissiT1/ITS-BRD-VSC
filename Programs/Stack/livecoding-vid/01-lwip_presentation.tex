
\documentclass{beamer}
\usetheme{Berlin}
\RequirePackage[utf8]{inputenc}
\RequirePackage[T2A]{fontenc}


\RequirePackage{amsfonts}

\usepackage{algorithmic}
\usepackage{mathptmx}
\usepackage[scaled=.90]{helvet}
\usepackage{courier}
\usepackage{pst-node}
\usepackage{pst-tree}

\usepackage{color}
\usepackage{colortbl}
\usepackage{pstricks}
\usepackage{verbatim}
\usepackage{alltt}
\usepackage{listings}
\usepackage{animate}
\usepackage{multimedia}


\title{Einführung in Git und Entwicklung mit lwIP-Stack}
\author{Martin}
\date{\today}

\lstset{
    basicstyle=\ttfamily\small,   % Schriftart für den Code
    keywordstyle=\color{blue},    % Schlüsselwörter in Blau
    commentstyle=\color{green},   % Kommentare in Grün
    stringstyle=\color{red},      % Strings in Rot
    showstringspaces=false,       % Leerzeichen in Strings nicht anzeigen
    breaklines=true,              % Zeilenumbruch bei langen Zeilen
    frame=single                  % Rahmen um den Code
}

\begin{document}

\begin{frame}
  \titlepage
\end{frame}

\begin{frame}{Agenda}
  \tableofcontents
\end{frame}

\section{Projektinitialisierung}
\begin{frame}[fragile]\frametitle{Projekt von GitHub herunterladen}
  \begin{itemize}
    \item Initialisierung der Submodule:
    \begin{verbatim}
      git clone https://github.com/...
      git submodule init
      git submodule update --recursive
    \end{verbatim}
    \item Alle Bibliotheken werden geladen
    \item Vorbereitetes lwIP-Projekt beinhaltet alle notwendigen Implementierungen
  \end{itemize}
\end{frame}

\section{Featureentwicklung}
\begin{frame}[fragile]\frametitle{PBranch für neues Feature anlegen}
  \begin{itemize}
    \item Neuen Branch erstellen:
    \begin{verbatim}
      git checkout -b stm
    \end{verbatim}
    \item Neues Projekt erstellen, Beispielprojekte nicht verändern
    \item Script „create new project“ nutzen
  \end{itemize}
\end{frame}

\begin{frame}{Projekt kompilieren und flashen}
  \begin{itemize}
    \item "Clean All" vor dem Kompilieren ausführen
    \item Projekt auf das Board flashen
    \item Wichtig: stabile Verbindungen sicherstellen (USB-Verlängerung)
  \end{itemize}
\end{frame}

\section{State Machine Implementierung}
\begin{frame}{State Machine entwickeln}
  \begin{itemize}
    \item State Machine für Task-Management implementieren
    \item Funktionszeiger und Zeithandling für Tasks
    \item Code soll erweiterbar sein
  \end{itemize}
\end{frame}

\begin{frame}{Verwendung von ChatGPT für Codegenerierung}
  \begin{itemize}
    \item Nutzung von ChatGPT zur Unterstützung bei der Codeerstellung
    \item Beispiel: State Machine mit Funktionszeigern für Tasks
    \item Anpassung des generierten Codes für Echtzeitfähigkeit
  \end{itemize}
\end{frame}

\section{Modulentwicklung}
\begin{frame}{Anlegen eines neuen Moduls}
  \begin{itemize}
    \item Neues Modul für LED-Steuerung erstellen
    \item Struktur von GPIO-Pins definieren
    \item Precompiler-Direktiven und Funktionsdeklarationen verwenden
  \end{itemize}
\end{frame}

\begin{frame}{Implementierung der LED-Steuerung}
  \begin{itemize}
    \item Implementierung der Funktion `ToggleGPIO`
    \item Funktionsaufruf im Task-Manager integrieren
    \item Debugging ohne printf, stattdessen LED-Feedback verwenden
  \end{itemize}
\end{frame}

\section{Projekt abschließen}
\begin{frame}[fragile]\frametitle{PCodeversionierung}
  \begin{itemize}
    \item Änderungen committen:
    \begin{verbatim}
      git add .
      git commit -m "Implemented LED control and state machine"
    \end{verbatim}
    \item Branches sauber verwalten
    \item Projektdateien und neue Module korrekt versionieren
  \end{itemize}
\end{frame}

\begin{frame}{Zusammenfassung}
  \begin{itemize}
    \item Projektinitialisierung und Submodule
    \item State Machine und Taskverwaltung
    \item Nutzung von ChatGPT zur Codeerstellung
    \item Modulentwicklung für GPIO-Steuerung
    \item Commit und Versionierung
  \end{itemize}
\end{frame}

\end{document}
